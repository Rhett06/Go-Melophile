\section{Conclusions}\label{sec:conclusions}

In this research we conducted an empirical experiment to measure the energy consumption of two music streaming applications Spotify and YouTube Music on the Android mobile device. Apart from the general comparison of energy consumed per app, we analyzed three energy-consuming factors to see if they have significant impacts on the energy consumption, namely: (1) connection type (\ie Wi-Fi streaming vs. downloaded file playing), (2) sound volume (\ie low, medium, and high), and (3) audio quality (\ie low, normal, high, very high). Our findings reveal that the energy consumption of Spotify significantly differentiates from YouTube Music with 17.2\% energy savings. In particular, the connection type influences the energy consumption in Spotify whereas the difference is minor in YouTube Music. Despite the significant impact on the consumed energy in Spotify, we deemed sound volume as a negligible issue, especially the low and the medium level. Lastly, audio quality is the dominant factor among all three to induce higher energy consumption for both applications. To summarize, this study provides guidance for users to extend their battery life and optimize towards energy consumption by adopting offline listening and avoiding high volume level and high audio quality.

The limitations of this research create opportunities for future study. Given the fact that the relatively short duration in each run may undermine the accuracy of energy estimation, we are planning to replicate the experiment with an extended time period of 10 minutes. Also, it should involve more music streaming applications to achieve better generalization. What’s more, other well-recognized energy-demanding factors like applications running as a foreground or background activity should be taken into consideration. It is a typical scenario that users have the music application running in the background with the screen locked. However, activating the app is expected to consume more energy due to the increasing uptime.

Finally, the replicated package is available for reproduction of our research as well as further investigation. 

\href{https://docs.google.com/spreadsheets/d/1w2QG47_3Y9-IXbPPSM2fw1bH0ZvWMfadzRdfwqmWBtI/edit?usp=sharing}{\textbf{Time Log}}