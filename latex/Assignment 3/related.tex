\section{Related Work}\label{sec:related}
In the work of Baek et al. \cite{baek2018energy} an energy efficiency grading system is developed to measure the energy consumption of mobile applications in the context of usage patterns that generalize the common features among interchangeable products. Rather than generating the accurate power usage information, the proposed system labels the energy efficiency grades for intuitive readability. The authors implement the system to evaluate the energy consumption of four leading music applications on Android devices in terms of three music play usage patterns including cached file play, streaming play, and combined play. Firstly, they measure the energy consumption of one music player running in different modes. Further, they introduce a new independent variable music genre (\ie classic, rock, and pop). Combined with music usage patterns, they document the energy consumption of all four applications. To conclude, downloaded play is more energy-greedy and rock music consumes more power in general. In contrast, our research focuses exclusively on energy consumption of music platforms, and does not intend to come up with general usage patterns that can be applied to different applications, not to mention a energy consumption grading system. Our study aims at conducting an in-depth empirical experiment to investigate possible power-hungry scenarios including Wi-Fi streaming vs. offline playing, {\color{blue} different sound volumes}, and different audio qualities. 

Nyman \cite{nyman2020estimating} carries out the research on Spotify to find proxy metrics that have the greatest impact on energy consumption and further determine potential relationships between measured metrics and energy consumption. The author examines three most power-consuming components (i.e., network, CPU, and memory) to construct linear models. The test cases are formalized based on typical user interactions, such as searching, navigation and playback. The author identifies three collinear pairs. They are user CPU and system CPU, sound and transmitted network, and received network bytes and written memory bytes. With the prevention of the use of all variables in the same pair above, the optimal model consists of transmitted network bytes, read and written memory bytes, and user CPU, which presents a 32.5\% increase compared to non-optimized ones. Differently from the aforementioned study, rather than building test suites based on common user behaviors shared with all kinds of applications, we narrow down to the playback scenario that is characterized as the fundamental feature of music platforms. 

Mehrotra et al. \cite{mehrotra2018analyse} proposes a framework that utilizes fuzzy clustering to classify mobile applications into low, medium, high categories with respect to power consumption patterns. The authors discover that the energy consumption of similar music applications varies significantly in terms of working environments. In particular, streaming and downloading consumes the highest energy whereas streaming without downloading falls in the ‘medium’ cluster. Likewise, switching off the LCD (Liquid Crystal Display) screen while listening to music downgrades the energy usage level from medium to low. Compared to this work, our study targets music applications and introduces additional independent variables like audio quality. Considering the relatively small dataset, the energy efficiency classification is out of scope.

Zhang et al. \cite{zhang2014impact} put forward a user-centric method to demonstrate that energy consumption varies dramatically depending on use cases and applications. The goal is to provide guidance for users to maximize the battery life as well as inform users the trade-offs they can make for the benefit of energy efficiency. For instance, when performing the same task, GUI (Graphical User Interface) music player is six times more energy-demanding than a command-line version. It is not surprising that GUI and music libraries incur more energy consumption. Working toward the common user-oriented goal, we narrow down to commercial music applications implemented with sophisticated features. Instead of taking unconfigurable GUI animation into consideration, we test the audio quality that is considered another important energy-consuming factor. 

In the research of Jimenez et al. \cite{jimenez2013integrating}, they propose a protocol modification to further integrate mobile devices into Spotify’s P2P network for the purpose of dynamically adapting to different devices. They compute the power consumption related to wireless network activities, namely Wi-Fi and 3G. In terms of the 3G network, they report that there is a long delay between radio state transitions, which increases power consumption. In contrast, Wi-Fi data transfer is much more energy-efficient because of PSM. Our user-driven study complements the developer-oriented work by Jimenez et al. Motivated by the general concerns that streaming may drain the battery, we measure the energy usage in both online (\ie Wi-Fi) and offline environments. 
% Describe here scientific papers similar to your experiment, both in terms of goal and methodology. 

% One paragraph for each paper (we expect about 5-8 papers to be discussed). Each paragraph contains: (i) a brief description of the related paper and (ii) a black-on-white description about how your experiment differs from the related paper.

%\textcolor{red}{Page limit: 1}