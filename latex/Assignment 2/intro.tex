\section{Introduction}
Given the prevalence of the Internet, the way users were able to listen to music tremendously changed. Before music streaming services gained popularity, music was typically delivered by radio and physical albums. In both cases, users have no or little control over the content \cite{hiller2017rise}. However, nowadays music streaming platforms like Spotify, Apple Music, and YouTube Music allow users to listen to music on demand from any location via various Internet-connected devices. What’s more, features like gigantic music libraries, personalized song recommendation, seamless synchronization across devices lead to these services reaching iconic product status. Apart from entertainment value, music applications also play an important role in social life{\color{blue}, e.g., they provide the opportunity for people to make connections with each other \cite{oyedele2018streaming}. Streaming services account for 65\% of the total music industry revenue in 2021, the eleventh consecutive year of growth, according to IFPI, the organization that reports the total recorded music industry worldwide \cite{3}.} 
%: They provide the opportunity for people to make connections with each other \cite{oyedele2018streaming}. %
By looking into the explosive growth of music streaming services, it is no exaggeration to say that streaming is the only future as it becomes the dominant format of music distribution. A recent research shows that the global music streaming market size was valued at USD 29.45 billion in 2021 \cite{4}.

Considering people’s fascination with music, applications tend to cause excessive battery usage and throttle the performance of devices. These may lead to users uninstalling and switching to competitors. Meanwhile, developers face technical challenges when attempting to improve hardware and/or software. On the one hand, the extreme complexity of semiconductor process technology has “slowed down” Moore’s Law, which further limits the development of chips \cite{waldrop2016chips}. Furthermore, lithium-ion batteries face the inherent “capacity fade” deficiency, which refers to the loss in discharge capacity with repeated use \cite{spotnitz2003simulation}. On the other hand, as Nathan P. Myhrvold says, software always pushes the performance boundary no matter how much improvement has been achieved in hardware. Therefore, how to optimize towards energy efficiency by changing the related settings becomes a critical task for end users {\color{blue}\cite{romaine1999invisible}}. 

Despite the long-standing problem of music players draining the battery, scarce work has focused on the energy consumption of music streaming applications. In order to investigate the energy consumption of music players quantitatively, we conduct experiments on the leading Android music applications Spotify\footnote{\label{note1}\href{ https://play.google.com/store/apps/details?id=com.spotify.music}{https://play.google.com/store/apps/details?id=com.spotify.music}} and YouTube Music\footnote{\label{note1}\href{ https://play.google.com/store/apps/details?id=com.google.android.apps.youtube.music}{https://play.google.com/store/apps/details?id=com.google.android.apps.youtube.music}}. Specifically, the study examines the representative scenarios and provides in-depth insights into potential triggers that result in steep battery decrease. In addition to comparing the general energy consumption of different applications, the result documents empirical evidence to identify which features are more power-hungry, and whether or not the same feature has different impacts across applications. The ultimate purpose of this study is to empower {\color{blue}Android phone }users to extend the life span of the battery. 

This research concentrates solely on the {\color{blue}energy} consumed by the music player running on Android devices. In other words, we play music via an internal speaker rather than Bluetooth to exclude the influences of external devices like headphones. To this extend, this paper makes the following contributions: 
\begin{itemize}
\item We provide a comprehensive comparison of the energy consumption between two dominant music streaming platforms Spotify and YouTube Music. 
\item We {\color{blue}measure} the impacts of various {\color{blue}streaming applications} settings on energy consumption, namely (1) Wi-Fi streaming or offline playing (\ie downloaded files), (2) whether or not the user directly runs the app (\ie foreground or background activity), and (3) audio quality (\ie low, normal, high, and very high). 
\end{itemize}

The remainder of this paper is structured as follows: Section 2 presents the related work and research questions are elaborated on Section 3. Section 4 and 5 report the design and execution of the experiment respectively. Followed by the main findings in Section 6, we discuss the potential limitations in Section 7. Section 8 analyzes threats to validity for our findings. Last but not the least, Section 9 concludes the paper. 

% This document represents a template of the final experiment report structure for the course \textit{Green Lab} at the Vrije Universiteit Amsterdam \cite{greenlab}.

% The experiment is conducted according to the guidelines by Wohlin and colleagues \cite{wohlin12}.

% The total length of this document must not exceed 15 pages, including references, appendixes, \etc

% In this section you have to describe (i) the domain (\eg mobile apps and their market) and the technologies relevant for understanding the rest of the document, (ii) the main motivation behind your experiment (the problem, here you can show examples via apps/tools screenshots, snippets of source code, \etc), (iii) what your experiment is about (hint of the solution), and (iii) what the developers will learn from the results of your experiment.  \ie

% \textcolor{red}{Page limit: 2}
